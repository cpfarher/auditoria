\documentclass[10pt,a4paper,final]{article}
%
\usepackage[utf8x]{inputenc}
\usepackage{ucs}
\usepackage{amsmath}
\usepackage{geometry}
\usepackage{anysize} % Soporte para el comando \marginsize
\usepackage{graphicx}
\usepackage{listings}
\usepackage{amsfonts}
\usepackage{amssymb}
\usepackage[spanish]{babel}
%%%%%%%%%%%%%%%%%%%%%%%%%%%%%%%%%%%%%%%%%%%%%%%%%
%%%%%%%%%%%%%%%%%%%%%%%%%%%%%%%%%%%%%%%%%%%%%%%%%
%%%%%%%%%%%%%%%%%%%%%%%%%%%%%%%%%%%%%%%%%%%%%%%%%
\marginsize{2cm}{2cm}{1.5cm}{1.5cm}
%
\begin{document}
\title{Auditoría Informática de Ing. Química}
\author{Fornal, Esteban, Garbarino, Juan Pablo, Pfarher Christian\\
\textit{Trabajo práctico final de ``Auditoría Informática'', II-FICH-UNL.}}
\markboth{Auditoría Informática: TRABAJO FINAL}{}
\date{\today}
\maketitle
%%%%%%%%%%%%%%%%%%%%%%%%%%%%%%%%%%%%%%%%%%%%%%%%%%%%%%%%%%%%%%%%%%%%%%%%%%%%%%%%%%%%%%%%%%%%%%%%%%
%%%%%%%%%%%%%%%%%%%%%%%%%%%%%%%%%%%%%%%%%%%%%%%%%%%%%%%%%%%%%%%%%%%%%%%%%%%%%%%%%%%%%%%%%%%%%%%%%%
%%%%%%%%%%%%%%%%%%%%%%%%%%%%%%%%%%%%%%%%%%%%%%%%%%%%%%%%%%%%%%%%%%%%%%%%%%%%%%%%%%%%%%%%%%%%%%%%%%
\newpage
\tableofcontents
%%%%%%%%%%%%%%%%%%%%%%%%%%%%%%%%%%%%%%%%%%%%%%%%%%%%%%%%%%%%%%%%%%%%%%%%%%%%%%%%%%%%%%%%%%%%%%%%%%
%%%%%%%%%%%%%%%%%%%%%%%%%%%%%%%%%%%%%%%%%%%%%%%%%%%%%%%%%%%%%%%%%%%%%%%%%%%%%%%%%%%%%%%%%%%%%%%%%%
%%%%%%%%%%%%%%%%%%%%%%%%%%%%%%%%%%%%%%%%%%%%%%%%%%%%%%%%%%%%%%%%%%%%%%%%%%%%%%%%%%%%%%%%%%%%%%%%%%
%\newpage
%\listoffigures % Índice de figuras
%%%%%%%%%%%%%%%%%%%%%%%%%%%%%%%%%%%%%%%%%%%%%%%%%%%%%%%%%%%%%%%%%%%%%%%%%%%%%%%%%%%%%%%%%%%%%%%%%%
%%%%%%%%%%%%%%%%%%%%%%%%%%%%%%%%%%%%%%%%%%%%%%%%%%%%%%%%%%%%%%%%%%%%%%%%%%%%%%%%%%%%%%%%%%%%%%%%%%
%%%%%%%%%%%%%%%%%%%%%%%%%%%%%%%%%%%%%%%%%%%%%%%%%%%%%%%%%%%%%%%%%%%%%%%%%%%%%%%%%%%%%%%%%%%%%%%%%%
\newpage
%%%%%%%%%%%%%%%%%%%%%%%%%%%%%%%%%%%%%%%%%%%%%%%%%%%%%%%%%%%%%%%%%%%%%%%%%%%%%%%%%%%%%%%%%%%%%%%%%%
%%%%%%%%%%%%%%%%%%%%%%%%%%%%%%%%%%%%%%%%%%%%%%%%%%%%%%%%%%%%%%%%%%%%%%%%%%%%%%%%%%%%%%%%%%%%%%%%%%
%%%%%%%%%%%%%%%%%%%%%%%%%%%%%%%%%%%%%%%%%%%%%%%%%%%%%%%%%%%%%%%%%%%%%%%%%%%%%%%%%%%%%%%%%%%%%%%%%%
\section{Introducción}
Presentar formalmente la empresa u organizacion donde se realizara el trabajo
\subsection{Organigrama de la organización}
%%%%%%%%%%%%%%%%%%%%%%%%%%%%%%%%%%%%%%%%%%%%%%%%%%%%%%%%%%%%%%%%%%%%%%%%%%%%%%%%%%%%%%%%%%%%%%%%%%
%%%%%%%%%%%%%%%%%%%%%%%%%%%%%%%%%%%%%%%%%%%%%%%%%%%%%%%%%%%%%%%%%%%%%%%%%%%%%%%%%%%%%%%%%%%%%%%%%%
%%%%%%%%%%%%%%%%%%%%%%%%%%%%%%%%%%%%%%%%%%%%%%%%%%%%%%%%%%%%%%%%%%%%%%%%%%%%%%%%%%%%%%%%%%%%%%%%%%
\section{Relevamiento}
A partir de este punto solo se describen aspectos que tienen que ver con la tecnologia
Identificar eo presentar el sector o Area de tecnologia
\subsection{Organigrama del área relevada}
    2.1) Organigrama -> funcionalmente de quien depende. (este organigrama esta vinculado con 1.1)
          Dotacion de recursos humanos (cant. para llevar adelante sus objetivos)
%%%%%%%%%%%%%%%%%%%%%%%%%%%%%%%%%%%%%%%%%%%%%%%%%%%%%%%%%%%%%%%%%%%%%%%%%%%%%%%%%%%%%%%%%%%%%%%%%%
%%%%%%%%%%%%%%%%%%%%%%%%%%%%%%%%%%%%%%%%%%%%%%%%%%%%%%%%%%%%%%%%%%%%%%%%%%%%%%%%%%%%%%%%%%%%%%%%%%
%%%%%%%%%%%%%%%%%%%%%%%%%%%%%%%%%%%%%%%%%%%%%%%%%%%%%%%%%%%%%%%%%%%%%%%%%%%%%%%%%%%%%%%%%%%%%%%%%%
\section{Estrategias Informáticas}
\subsection{Estrategias a largo plazo}
proyectos o planes
\subsection{Estrategias a mediano plazo}
\subsection{Estrategias a corto plazo}

\section{Hardware}
en general..
\section{Software}
Software (Gral.)
\subsection{Software Base}
s.o., gestor BD, antivirus, leng. programación, tiene licencias o no? soft libre?
\subsection{Software Aplicativo}
dirigido a satisfacer algún aspecto puntual del negocio (identificarlos-> nombres y /o para que esta)
\section{Esquema de comunicaciones}
 o telecomunicaciones -> Redes, mostrar esquema o layout, vinculo entre sectores
\section{Estructura de procesamiento}
: tiene que ver con la forma en que una empresa trabaja con tecnologia (tareas atipicas no rutinarias, en horarios espciales...)
\section{Normativas internas}
 (Iso, normas internas de seguridad, etc.) Decir si no hay normas internas o si no han asumido normas internas.
\section{Continuidad del procesamiento}
 (Situaciones inesperadas que hagan que la organizacion tenga que adoptar alguna medida para seguir trabajando).-> catastrofes ambientales, servidores que no funcionen, redes o sistemas caidos (plan de contingencia). Esta garantizado? Como?
\section{Seguridad lógica}
: aspectos no palpables desde el punto de vista de la tecnologia. Privilegios de usuarios, puesto de trabajo -> Quien puede acceder? Backups -> Supervisado? Resguardo. hay controles respecto a esos datos?
\section{Relación con terceros}
\section{Políticas de selección y entrenamiento de personal y /o recursos humanos}
 -> por recomendaciones?, acuerdo con universidad?,avisos del diario? capacita a los recursos humanos?, los promocionamos para que escale en la empresa?
\section{Identificación de problemas, necesidades, incertidumbres existentes en la organización vinculadas con las tecnologías}. Necesidades insastisfechas, dudas con la tecnologia?->parte importantisima

\section{etapa 2}
En base a lo que surja en el punto 13. De todos los problemas reales o potenciales la empresa tiene interes en buscar soluciones a ciertos problemas?
Aprox. los problemas.
\end{document}